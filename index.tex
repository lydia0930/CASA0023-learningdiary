% Options for packages loaded elsewhere
\PassOptionsToPackage{unicode}{hyperref}
\PassOptionsToPackage{hyphens}{url}
\PassOptionsToPackage{dvipsnames,svgnames,x11names}{xcolor}
%
\documentclass[
  letterpaper,
  DIV=11,
  numbers=noendperiod]{scrreprt}

\usepackage{amsmath,amssymb}
\usepackage{iftex}
\ifPDFTeX
  \usepackage[T1]{fontenc}
  \usepackage[utf8]{inputenc}
  \usepackage{textcomp} % provide euro and other symbols
\else % if luatex or xetex
  \usepackage{unicode-math}
  \defaultfontfeatures{Scale=MatchLowercase}
  \defaultfontfeatures[\rmfamily]{Ligatures=TeX,Scale=1}
\fi
\usepackage{lmodern}
\ifPDFTeX\else  
    % xetex/luatex font selection
\fi
% Use upquote if available, for straight quotes in verbatim environments
\IfFileExists{upquote.sty}{\usepackage{upquote}}{}
\IfFileExists{microtype.sty}{% use microtype if available
  \usepackage[]{microtype}
  \UseMicrotypeSet[protrusion]{basicmath} % disable protrusion for tt fonts
}{}
\makeatletter
\@ifundefined{KOMAClassName}{% if non-KOMA class
  \IfFileExists{parskip.sty}{%
    \usepackage{parskip}
  }{% else
    \setlength{\parindent}{0pt}
    \setlength{\parskip}{6pt plus 2pt minus 1pt}}
}{% if KOMA class
  \KOMAoptions{parskip=half}}
\makeatother
\usepackage{xcolor}
\setlength{\emergencystretch}{3em} % prevent overfull lines
\setcounter{secnumdepth}{5}
% Make \paragraph and \subparagraph free-standing
\ifx\paragraph\undefined\else
  \let\oldparagraph\paragraph
  \renewcommand{\paragraph}[1]{\oldparagraph{#1}\mbox{}}
\fi
\ifx\subparagraph\undefined\else
  \let\oldsubparagraph\subparagraph
  \renewcommand{\subparagraph}[1]{\oldsubparagraph{#1}\mbox{}}
\fi

\usepackage{color}
\usepackage{fancyvrb}
\newcommand{\VerbBar}{|}
\newcommand{\VERB}{\Verb[commandchars=\\\{\}]}
\DefineVerbatimEnvironment{Highlighting}{Verbatim}{commandchars=\\\{\}}
% Add ',fontsize=\small' for more characters per line
\usepackage{framed}
\definecolor{shadecolor}{RGB}{241,243,245}
\newenvironment{Shaded}{\begin{snugshade}}{\end{snugshade}}
\newcommand{\AlertTok}[1]{\textcolor[rgb]{0.68,0.00,0.00}{#1}}
\newcommand{\AnnotationTok}[1]{\textcolor[rgb]{0.37,0.37,0.37}{#1}}
\newcommand{\AttributeTok}[1]{\textcolor[rgb]{0.40,0.45,0.13}{#1}}
\newcommand{\BaseNTok}[1]{\textcolor[rgb]{0.68,0.00,0.00}{#1}}
\newcommand{\BuiltInTok}[1]{\textcolor[rgb]{0.00,0.23,0.31}{#1}}
\newcommand{\CharTok}[1]{\textcolor[rgb]{0.13,0.47,0.30}{#1}}
\newcommand{\CommentTok}[1]{\textcolor[rgb]{0.37,0.37,0.37}{#1}}
\newcommand{\CommentVarTok}[1]{\textcolor[rgb]{0.37,0.37,0.37}{\textit{#1}}}
\newcommand{\ConstantTok}[1]{\textcolor[rgb]{0.56,0.35,0.01}{#1}}
\newcommand{\ControlFlowTok}[1]{\textcolor[rgb]{0.00,0.23,0.31}{#1}}
\newcommand{\DataTypeTok}[1]{\textcolor[rgb]{0.68,0.00,0.00}{#1}}
\newcommand{\DecValTok}[1]{\textcolor[rgb]{0.68,0.00,0.00}{#1}}
\newcommand{\DocumentationTok}[1]{\textcolor[rgb]{0.37,0.37,0.37}{\textit{#1}}}
\newcommand{\ErrorTok}[1]{\textcolor[rgb]{0.68,0.00,0.00}{#1}}
\newcommand{\ExtensionTok}[1]{\textcolor[rgb]{0.00,0.23,0.31}{#1}}
\newcommand{\FloatTok}[1]{\textcolor[rgb]{0.68,0.00,0.00}{#1}}
\newcommand{\FunctionTok}[1]{\textcolor[rgb]{0.28,0.35,0.67}{#1}}
\newcommand{\ImportTok}[1]{\textcolor[rgb]{0.00,0.46,0.62}{#1}}
\newcommand{\InformationTok}[1]{\textcolor[rgb]{0.37,0.37,0.37}{#1}}
\newcommand{\KeywordTok}[1]{\textcolor[rgb]{0.00,0.23,0.31}{#1}}
\newcommand{\NormalTok}[1]{\textcolor[rgb]{0.00,0.23,0.31}{#1}}
\newcommand{\OperatorTok}[1]{\textcolor[rgb]{0.37,0.37,0.37}{#1}}
\newcommand{\OtherTok}[1]{\textcolor[rgb]{0.00,0.23,0.31}{#1}}
\newcommand{\PreprocessorTok}[1]{\textcolor[rgb]{0.68,0.00,0.00}{#1}}
\newcommand{\RegionMarkerTok}[1]{\textcolor[rgb]{0.00,0.23,0.31}{#1}}
\newcommand{\SpecialCharTok}[1]{\textcolor[rgb]{0.37,0.37,0.37}{#1}}
\newcommand{\SpecialStringTok}[1]{\textcolor[rgb]{0.13,0.47,0.30}{#1}}
\newcommand{\StringTok}[1]{\textcolor[rgb]{0.13,0.47,0.30}{#1}}
\newcommand{\VariableTok}[1]{\textcolor[rgb]{0.07,0.07,0.07}{#1}}
\newcommand{\VerbatimStringTok}[1]{\textcolor[rgb]{0.13,0.47,0.30}{#1}}
\newcommand{\WarningTok}[1]{\textcolor[rgb]{0.37,0.37,0.37}{\textit{#1}}}

\providecommand{\tightlist}{%
  \setlength{\itemsep}{0pt}\setlength{\parskip}{0pt}}\usepackage{longtable,booktabs,array}
\usepackage{calc} % for calculating minipage widths
% Correct order of tables after \paragraph or \subparagraph
\usepackage{etoolbox}
\makeatletter
\patchcmd\longtable{\par}{\if@noskipsec\mbox{}\fi\par}{}{}
\makeatother
% Allow footnotes in longtable head/foot
\IfFileExists{footnotehyper.sty}{\usepackage{footnotehyper}}{\usepackage{footnote}}
\makesavenoteenv{longtable}
\usepackage{graphicx}
\makeatletter
\def\maxwidth{\ifdim\Gin@nat@width>\linewidth\linewidth\else\Gin@nat@width\fi}
\def\maxheight{\ifdim\Gin@nat@height>\textheight\textheight\else\Gin@nat@height\fi}
\makeatother
% Scale images if necessary, so that they will not overflow the page
% margins by default, and it is still possible to overwrite the defaults
% using explicit options in \includegraphics[width, height, ...]{}
\setkeys{Gin}{width=\maxwidth,height=\maxheight,keepaspectratio}
% Set default figure placement to htbp
\makeatletter
\def\fps@figure{htbp}
\makeatother

\KOMAoption{captions}{tableheading}
\makeatletter
\@ifpackageloaded{bookmark}{}{\usepackage{bookmark}}
\makeatother
\makeatletter
\@ifpackageloaded{caption}{}{\usepackage{caption}}
\AtBeginDocument{%
\ifdefined\contentsname
  \renewcommand*\contentsname{Table of contents}
\else
  \newcommand\contentsname{Table of contents}
\fi
\ifdefined\listfigurename
  \renewcommand*\listfigurename{List of Figures}
\else
  \newcommand\listfigurename{List of Figures}
\fi
\ifdefined\listtablename
  \renewcommand*\listtablename{List of Tables}
\else
  \newcommand\listtablename{List of Tables}
\fi
\ifdefined\figurename
  \renewcommand*\figurename{Figure}
\else
  \newcommand\figurename{Figure}
\fi
\ifdefined\tablename
  \renewcommand*\tablename{Table}
\else
  \newcommand\tablename{Table}
\fi
}
\@ifpackageloaded{float}{}{\usepackage{float}}
\floatstyle{ruled}
\@ifundefined{c@chapter}{\newfloat{codelisting}{h}{lop}}{\newfloat{codelisting}{h}{lop}[chapter]}
\floatname{codelisting}{Listing}
\newcommand*\listoflistings{\listof{codelisting}{List of Listings}}
\makeatother
\makeatletter
\makeatother
\makeatletter
\@ifpackageloaded{caption}{}{\usepackage{caption}}
\@ifpackageloaded{subcaption}{}{\usepackage{subcaption}}
\makeatother
\ifLuaTeX
  \usepackage{selnolig}  % disable illegal ligatures
\fi
\usepackage{bookmark}

\IfFileExists{xurl.sty}{\usepackage{xurl}}{} % add URL line breaks if available
\urlstyle{same} % disable monospaced font for URLs
\hypersetup{
  pdftitle={CASA0023 Learning Diary},
  pdfauthor={Yundi LIU},
  colorlinks=true,
  linkcolor={blue},
  filecolor={Maroon},
  citecolor={Blue},
  urlcolor={Blue},
  pdfcreator={LaTeX via pandoc}}

\title{CASA0023 Learning Diary}
\author{Yundi LIU}
\date{2025-01-25}

\begin{document}
\maketitle

\renewcommand*\contentsname{Table of contents}
{
\hypersetup{linkcolor=}
\setcounter{tocdepth}{2}
\tableofcontents
}
\bookmarksetup{startatroot}

\chapter{CASA0023 Learning Diary}\label{casa0023-learning-diary}

\bookmarksetup{startatroot}

\chapter{Welcome to CASA0023 Learning
Diary}\label{welcome-to-casa0023-learning-diary}

This is my learning diary for CASA0023. It contains weekly reflections
and notes on various topics.

\begin{itemize}
\tightlist
\item
  \textbf{Week 1:} Introduction to remote sensing
\item
  \textbf{Week 2:} Presentation on sensors
\item
  \textbf{Week 3:} Corrections and feedback
\item
  \textbf{Week 4:} Policy discussion
\item
  \textbf{Week 6:} Google Earth Engine analysis
\item
  \textbf{Week 7:} Classification I
\item
  \textbf{Week 8:} Classification II
\item
  \textbf{Week 9:} Synthetic Aperture Radar (SAR)
\end{itemize}

\textbf{References and additional materials} are included at the end of
each note.

\bookmarksetup{startatroot}

\chapter{Introduction}\label{introduction}

\bookmarksetup{startatroot}

\chapter{An Introduction to Remote
Sensing}\label{an-introduction-to-remote-sensing}

\#\#Summary

\#\#\#Definition of remote sensing

Remote sensing is the science of obtaining information about objects or
areas from a distance using sensors that detect electromagnetic
radiation. The electromagnetic spectrum is crucial to understanding
which wavelengths are useful for different applications. The key diagram
to include would be the electromagnetic spectrum chart showing different
wavelengths and their applications in remote sensing.

\#\#\#Resolution Types in Remote Sensing

The four essential resolution types determine what can be detected:

\begin{longtable}[]{@{}
  >{\raggedright\arraybackslash}p{(\columnwidth - 2\tabcolsep) * \real{0.3056}}
  >{\raggedright\arraybackslash}p{(\columnwidth - 2\tabcolsep) * \real{0.6944}}@{}}
\toprule\noalign{}
\begin{minipage}[b]{\linewidth}\raggedright
Type
\end{minipage} & \begin{minipage}[b]{\linewidth}\raggedright
Definition
\end{minipage} \\
\midrule\noalign{}
\endhead
\bottomrule\noalign{}
\endlastfoot
Spatial resolution & The size of one pixel in ground units \\
Spectral resolution & Number and width of spectral bands \\
Temporal resolution & How often data is collected \\
Radiometric resolution & Sensitivity to differences in reflected/emitted
energy \\
\end{longtable}

It Includes the comparative images showing the same area captured at
different spatial resolutions (e.g., 30m vs.~10m) to illustrate how
detail changes.

\subsection{Satellite Systems and Their
Applications}\label{satellite-systems-and-their-applications}

Major satellite programs like Landsat (historical importance since 1972)
and Sentinel (newer European missions) provide different capabilities
for urban monitoring. The lecture highlights specifications including
revisit time, spatial resolution, and spectral bands. Include the
satellite timeline diagram showing the evolution of Earth observation
satellites and their operational periods.

\subsection{Urban Remote Sensing
Applications}\label{urban-remote-sensing-applications}

Remote sensing enables monitoring urban environments through
applications like land use classification, urban heat island detection,
and change detection over time. The thermal imagery showing urban heat
islands and comparison images of urban growth over decades would be
essential visuals.

\begin{center}\rule{0.5\linewidth}{0.5pt}\end{center}

\section{Applications}\label{applications}

In my research on remote sensing applications, I have found that these
technologies provide tools for understanding the urban environment.

How land use mapping is changing urban planning is one such application.
By analyzing imagery over time, we can track patterns of sprawl and even
identify informal settlements that traditional surveys might miss. This
capability is important for rapidly growing cities facing housing
challenges. Advanced machine learning techniques like object-based image
analysis (OBIA) now enable automatic classification of urban morphology
with accuracy exceeding 85\%, allowing planners to quantify not just
expansion but also densification processes.

In addition to this, urban heat island analysis applications are also
one of the options as climate issues become more serious. I find it
compelling that heat sensors can accurately identify hot spots that need
intervention. NASA Earth Observatory documents a clear pattern of
temperature differences of up to 10°C between urban and rural areas
(NASA Earth Observatory, 2021). This could change the way we approach
urban greening initiatives. Thermal infrared bands on satellites like
Landsat 8 (bands 10-11) and ECOSTRESS provide surface temperature data
at resolutions fine enough to correlate UHI intensity with specific
urban features and materials.

Finally, remote sensing appears to have the most significant impact in
disaster management applications. Its ability to provide timely
information throughout the disaster cycle - from vulnerability mapping
to response guidance and damage assessment - can save countless lives
and improve recovery efforts. For example, floods, hill fires, and crop
damage can all be assessed and predicted accordingly. Synthetic Aperture
Radar (SAR) sensors like those on Sentinel-1 have revolutionized
disaster monitoring by providing cloud-penetrating imagery day or night,
enabling near real-time flood extent mapping even during severe weather
conditions when optical sensors fail.

\includegraphics{index_files/mediabag/denmark_oli_1985-201.png} The
relationship between green space and relative mental health
\href{https://earthobservatory.nasa.gov/images/145305/summer-heat-shifts-to-europe}{Source:
NASA Earth Observatory}

\begin{center}\rule{0.5\linewidth}{0.5pt}\end{center}

\section{Reflection}\label{reflection}

As I engaged with the fundamentals of remote sensing, several thoughtful
connections emerged about this technology's place in science and
society.

The evolution of Earth observation technology is remarkable - from early
Landsat missions to today's specialized satellite constellations. This
progression mirrors other scientific fields where technological
advancement has transformed understanding, with remote sensing uniquely
turning our technological gaze back toward Earth itself.

The concept of different resolutions highlighted an important
philosophical point about observation. Our choice of resolution
determines what patterns become visible - broad land cover at coarse
resolution, individual buildings at finer scales. This reinforces that
all scientific observation involves choices about scale that
fundamentally shape understanding.

The democratization of satellite imagery particularly interests me. What
was once the exclusive domain of governments and corporations is
increasingly accessible, creating new possibilities for citizen science
and community monitoring. This shift raises important questions about
who gets to observe our environment and whose perspectives inform the
resulting narratives.

I found myself contemplating how remote sensing changes our relationship
with cities. The bird's-eye perspective offers insights unavailable to
most urban dwellers yet differs significantly from lived experiences at
street level. How might these different ways of knowing be integrated
for more comprehensive urban understanding?

Moving forward in this course, I'm particularly interested in exploring
how remote sensing data intersects with local knowledge and qualitative
understanding of places. The most valuable insights likely emerge when
technological observation and human experience are brought into
thoughtful conversation rather than treated as separate domains.

\bookmarksetup{startatroot}

\chapter{Presentation}\label{presentation}

\bookmarksetup{startatroot}

\chapter{Presentation}\label{presentation-1}

These are the introduction slides for the sensor ASTER(Advanced
Spaceborne Thermal Emission and Reflection Radiometer)

use this
\href{https://lydia0930.github.io/CASA0023xaringan/w2-slides.html}{Link}

\bookmarksetup{startatroot}

\chapter{Remote Sensing Data \&
Corrections}\label{remote-sensing-data-corrections}

\bookmarksetup{startatroot}

\chapter{Remote Sensing Data \&
Corrections}\label{remote-sensing-data-corrections-1}

\section{Summary}\label{summary}

In this lecture, I learned about the fundamentals of remote sensing,
focusing on data acquisition, sensor technologies, data processing, and
their applications in urban and environmental studies. The session was
divided into two main parts: image correction and data mosaicing and
enhancement. Below are the key takeaways:

\subsection{Sensor Types and
Mechanisms}\label{sensor-types-and-mechanisms}

Remote sensing sensors are at the core of remote sensing technology, and
different types of sensors employ different imaging mechanisms. The
following table compares two common types of sensors:

\begin{longtable}[]{@{}
  >{\raggedright\arraybackslash}p{(\columnwidth - 4\tabcolsep) * \real{0.2361}}
  >{\raggedright\arraybackslash}p{(\columnwidth - 4\tabcolsep) * \real{0.4861}}
  >{\raggedright\arraybackslash}p{(\columnwidth - 4\tabcolsep) * \real{0.2778}}@{}}
\toprule\noalign{}
\begin{minipage}[b]{\linewidth}\raggedright
Sensor Type
\end{minipage} & \begin{minipage}[b]{\linewidth}\raggedright
Working Principle
\end{minipage} & \begin{minipage}[b]{\linewidth}\raggedright
Pros \& Cons
\end{minipage} \\
\midrule\noalign{}
\endhead
\bottomrule\noalign{}
\endlastfoot
Whisk broom (cross-track scanner) & Uses a rotating mirror to reflect
light onto a single detector, covering a large area & Wide coverage, but
moving parts can wear out \\
Push broom (along-track scanner) & Uses detector arrays for faster and
more efficient imaging & High efficiency, but requires precise
calibration \\
MSS (Multispectral Scanner) & Uses a digital approach to capture
multispectral data (visible + near-infrared) & Provides better data
continuity and accuracy \\
\end{longtable}

Through these different sensor technologies, remote sensing systems are
able to capture various types of land feature data based on specific
needs.

\subsection{Data Correction Process}\label{data-correction-process}

Satellite imagery often requires correction before it can be used for
analysis. The following diagram shows the typical correction process:

\begin{itemize}
\item
  Start with a Raw Satellite Image.
\item
  Perform Geometric Correction to fix distortions caused by sensor
  motion or the Earth's curvature.
\item
  Apply Atmospheric Correction to remove haze or scattering effects, for
  example, using dark object subtraction.
\item
  Proceed with Terrain Correction to adjust for elevation differences,
  such as through orthorectification.
\end{itemize}

Data correction is a crucial first step to ensure the accuracy and
usability of remote sensing imagery. By applying geometric correction,
atmospheric correction, and terrain correction, we can eliminate
external disturbances and improve the precision of the data.

\subsection{Data Mosaicing and
Enhancement}\label{data-mosaicing-and-enhancement}

Mosaicing and enhancement techniques are vital for improving remote
sensing imagery, especially for urban and environmental studies.
However, these techniques must be chosen based on the specific needs of
the study. Standard mosaicing methods may not always be suitable for
satellite imagery, as there can be discrepancies in lighting and
radiometric calibration between images.

\begin{longtable}[]{@{}
  >{\raggedright\arraybackslash}p{(\columnwidth - 4\tabcolsep) * \real{0.2361}}
  >{\raggedright\arraybackslash}p{(\columnwidth - 4\tabcolsep) * \real{0.3611}}
  >{\raggedright\arraybackslash}p{(\columnwidth - 4\tabcolsep) * \real{0.4028}}@{}}
\toprule\noalign{}
\begin{minipage}[b]{\linewidth}\raggedright
Enhancement Method
\end{minipage} & \begin{minipage}[b]{\linewidth}\raggedright
Description
\end{minipage} & \begin{minipage}[b]{\linewidth}\raggedright
Challenges in Urban Applications
\end{minipage} \\
\midrule\noalign{}
\endhead
\bottomrule\noalign{}
\endlastfoot
\textbf{Mosaicing} & Combines multiple images into one seamless image &
Standard methods may cause radiometric inconsistencies \\
\textbf{Contrast Enhancement} & Improves clarity by enhancing feature
contrast & May over-complicate images, hindering research \\
\textbf{Creating New Datasets} & Uses band composition/feature
extraction & High computational load and analysis complexity \\
\end{longtable}

These enhancement methods can improve image contrast and clarity,
helping to analyze land cover changes more effectively. However, whether
added complexity truly benefits the research objectives needs careful
consideration.

\begin{center}\rule{0.5\linewidth}{0.5pt}\end{center}

\section{Applications}\label{applications-1}

Landsat satellites, providing over 50 years of Earth observation data
since 1972, have had a profound impact on monitoring urban growth,
deforestation, temperature changes, and more.

\subsection{Urban Growth \& Land Use
Mapping}\label{urban-growth-land-use-mapping}

One of the most powerful applications of remote sensing is its ability
to monitor urban growth and land use changes over time. Satellites like
Landsat and Sentinel-2 have been invaluable in tracking the
transformation of landscapes, especially as urban areas expand. The
integration of Landsat with other datasets, like Sentinel-2 through the
Dynamic World project, enables near-real-time land cover mapping, which
is crucial for monitoring urbanization, deforestation, and the spread of
infrastructure. For example, through the combination of these datasets,
researchers can track the extent of urban sprawl, assess changes in
built-up areas, and even identify shifts in the types of land cover
(e.g., from agriculture to residential). This kind of information is
vital for city planners and policymakers in managing sustainable urban
growth and mitigating the environmental impact of development.
Additionally, urban heat islands, a key environmental concern in cities,
can be monitored using thermal infrared data, which helps identify
regions that are overheating due to dense urbanization and limited
vegetation.

\subsection{Environmental Monitoring \& Disaster
Response}\label{environmental-monitoring-disaster-response}

Remote sensing also plays a critical role in environmental monitoring,
particularly in disaster response and management. Landsat and Sentinel-2
satellites have been used extensively to monitor wildfires, floods, and
changes in ecosystems. For example, Landsat's long-term data series
allows for the analysis of wildfire spread and recovery over decades.
These data sets provide valuable insights into the impact of fire on
vegetation and soil, and how these ecosystems recover over time.
Similarly, Sentinel-2's high temporal resolution makes it an excellent
tool for monitoring environmental changes more frequently, such as
detecting the early stages of flooding or tracking the aftermath ofS
extreme weather events. Remote sensing also supports vegetation health
monitoring by detecting anomalies in plant reflectance, particularly
using the near-infrared spectrum, which is sensitive to plant stress.
This capability is essential for agriculture, forestry, and conservation
efforts, enabling early intervention when vegetation shows signs of
degradation, drought, or pest infestations.

\begin{center}\rule{0.5\linewidth}{0.5pt}\end{center}

\section{Reflection}\label{reflection-1}

This week's practical exercise on atmospheric correction provided
invaluable hands-on experience that deepened my understanding of the
challenges involved in working with real-world remote sensing data. By
applying the Dark Object Subtraction (DOS) method and exploring more
advanced techniques like the 6S model, I developed a new appreciation
for the delicate balance researchers must strike between accuracy and
practicality.

The most enlightening moment came when I compared my corrected NDVI
values with field measurements. While the DOS method clearly enhanced my
results compared to using raw data, I observed significant variations in
urban areas, where identifying true ``dark'' pixels proved difficult.
This highlighted a crucial limitation I had not fully considered before
-- preprocessing methods that rely on specific assumptions might work
well in certain environments but fail in others.

What surprised me most was realizing the profound impact these technical
decisions can have on research outcomes. A seemingly minor choice of
correction method could potentially influence conclusions about
vegetation health or urban heat patterns. This made me wonder how many
published studies might be affected by similar preprocessing
limitations, which may not always be clearly explained in the methods
sections.

Additionally, the exercise made me reflect on the resource constraints
researchers face in real-world situations. While I learned about
theoretically superior methods like the 6S model, I now understand why
many researchers opt for simpler approaches. The time, data
requirements, and computational costs associated with advanced methods
can be prohibitive, particularly for students or small research teams.

Overall, data correction and enhancement are essential steps in
improving the quality of remote sensing data, enabling more accurate
monitoring of urban development and environmental changes. However,
increasing the complexity of imagery does not always enhance the
research value, and its benefits must be carefully considered in
relation to the specific research goals.

\bookmarksetup{startatroot}

\chapter{Policy}\label{policy}

\bookmarksetup{startatroot}

\chapter{Policy}\label{policy-1}

Week 4 focused on the applications of remote sensing to policy, and I
made a case study in Singapore.

\section{Summary}\label{summary-1}

\subsection{City Summary}\label{city-summary}

Singapore is a highly urbanized city-state located at the southern tip
of the Malay Peninsula in Southeast Asia. With a land area of just 728
km² and a population of 5.9 million, Singapore is one of the world's
most densely populated countries. The nation faces unique challenges due
to its limited land resources, including food security concerns as it
imports over 90\% of its food supply, vulnerability to climate change
effects like sea-level rise, and urban heat island intensification due
to its dense urban development.

\subsection{Policy Summary}\label{policy-summary}

\begin{enumerate}
\def\labelenumi{\arabic{enumi}.}
\item
  \textbf{Global Policy Overview}:

  Sustainable Development Goals (SDGs) Singapore has committed to
  implementing the United Nations' Sustainable Development Goals (SDGs),
  with particular emphasis on SDG 11 (Sustainable Cities and
  Communities) and SDG 13 (Climate Action). As a highly urbanized nation
  with limited natural resources, Singapore has prioritized creating a
  sustainable, resilient urban environment.
\item
  \textbf{Singapore Green Plan 2030} (Singapore Ministry of
  Sustainability and the Environment, 2021)

  The Singapore Green Plan 2030, launched by the government in 2021,
  serves as a national roadmap toward sustainable development and
  net-zero emissions. The plan integrates remote sensing technology to
  monitor environmental changes, track policy implementation, and
  evaluate effectiveness across various domains. Singapore's Urban
  Redevelopment Authority (URA) and National Environment Agency (NEA)
  have partnered with research institutions to employ Earth observation
  data for urban planning, environmental monitoring, and climate
  adaptation. This partnership leverages remote sensing to map urban
  heat islands, monitor green space development, and assess coastal
  changes to inform policy decisions and future planning.

  \begin{longtable}[]{@{}
    >{\raggedright\arraybackslash}p{(\columnwidth - 2\tabcolsep) * \real{0.4306}}
    >{\raggedright\arraybackslash}p{(\columnwidth - 2\tabcolsep) * \real{0.5694}}@{}}
  \toprule\noalign{}
  \begin{minipage}[b]{\linewidth}\raggedright
  Goal
  \end{minipage} & \begin{minipage}[b]{\linewidth}\raggedright
  Target
  \end{minipage} \\
  \midrule\noalign{}
  \endhead
  \bottomrule\noalign{}
  \endlastfoot
  Increasing urban green cover & From \textbf{47\% to 50\%} by 2030 \\
  Green spaces and trees & \textbf{1,000 hectares} of green spaces,
  \textbf{one million additional trees} \\
  Local food production & \textbf{20\%} of Singapore's total food demand
  by 2030 \\
  Waste reduction & \textbf{30\%} reduction in waste sent to landfill by
  2030 \\
  Solar energy expansion & \textbf{8,000 hectares} of solar panels on
  rooftops and reservoirs by 2030 \\
  Climate resilience & Strategies against \textbf{sea-level rise and
  extreme weather events} \\
  Water resource management & Sustainable management through
  \textbf{catchment management} \\
  \end{longtable}
\end{enumerate}

\includegraphics[width=0.7\textwidth,height=\textheight]{index_files/mediabag/image_2022_12_09T06_.png}

\href{https://www.greenplan.gov.sg/}{Source:Singapore greenplan}

\begin{center}\rule{0.5\linewidth}{0.5pt}\end{center}

\section{Application}\label{application}

This section identifies how remotely sensed data could be used to
contribute to Singapore's policy goals, answering the question:
\textbf{`How could the data be applied to solve the policy challenge?'}

\subsection{Data Sources}\label{data-sources}

Singapore employs multiple remote sensing platforms to support its Green
Plan implementation:

\begin{itemize}
\tightlist
\item
  \textbf{Satellite Imagery:}

  \begin{itemize}
  \tightlist
  \item
    \textbf{Sentinel-2}: Provides 10m resolution multispectral data to
    monitor urban vegetation changes.
  \item
    \textbf{Landsat data}: Offers a historical perspective for long-term
    change analysis.
  \item
    \textbf{Thermal Mapping}:

    \begin{itemize}
    \tightlist
    \item
      Landsat 8/9 thermal bands and ECOSTRESS data identify urban heat
      islands.
    \item
      Temperature readings are accurate to within 1°C.
    \end{itemize}
  \end{itemize}
\item
  \textbf{Unmanned Aerial Vehicles (UAVs):}

  \begin{itemize}
  \tightlist
  \item
    Provides ultra-high resolution (sub-meter) imagery.
  \item
    Used for monitoring specific urban projects, tree health in parks,
    and rooftop gardens.
  \end{itemize}
\item
  \textbf{Aerial LiDAR Surveys:}

  \begin{itemize}
  \tightlist
  \item
    Creates detailed 3D models of the urban environment.
  \item
    Essential for vegetation structure analysis and building-level
    energy studies.
  \end{itemize}
\end{itemize}

\subsection{Techniques}\label{techniques}

\begin{itemize}
\tightlist
\item
  \textbf{Urban Heat Island Monitoring}
\end{itemize}

\begin{enumerate}
\def\labelenumi{\arabic{enumi}.}
\tightlist
\item
  Acquire thermal infrared imagery from satellites and aerial platforms.
\item
  Calibrate and correct imagery for atmospheric effects.
\item
  Generate surface temperature maps across the urban area.
\item
  Identify heat island hotspots and correlate with urban features.
\item
  Track temperature changes in relation to green infrastructure
  implementation.
\item
  Develop targeted cooling strategies for hotspot areas.
\end{enumerate}

\begin{itemize}
\tightlist
\item
  \textbf{Green Space Assessment}
\end{itemize}

\begin{enumerate}
\def\labelenumi{\arabic{enumi}.}
\tightlist
\item
  Acquire high-resolution multispectral imagery.
\item
  Calculate vegetation indices (\textbf{NDVI, EVI}) to quantify
  greenery.
\item
  Classify urban vegetation types (\textbf{tree canopy, shrubs, grass}).
\item
  Monitor changes in green cover over time.
\item
  Evaluate effectiveness of tree-planting initiatives.
\item
  Identify priority areas for new green space development.
\end{enumerate}

\begin{itemize}
\tightlist
\item
  \textbf{Coastal Change Monitoring}
\end{itemize}

\begin{enumerate}
\def\labelenumi{\arabic{enumi}.}
\tightlist
\item
  Acquire satellite and aerial imagery of coastal areas.
\item
  Use \textbf{SAR imagery} for high-precision elevation measurements.
\item
  Detect changes in shoreline position and elevation.
\item
  Monitor effectiveness of coastal protection measures.
\item
  Model areas vulnerable to sea-level rise.
\item
  Develop adaptive strategies for coastal infrastructure.
\end{enumerate}

\begin{itemize}
\tightlist
\item
  \textbf{Urban Agriculture Mapping}
\end{itemize}

\begin{enumerate}
\def\labelenumi{\arabic{enumi}.}
\tightlist
\item
  Identify existing and potential rooftop farming locations.
\item
  Monitor productivity of urban farms with multispectral imagery.
\item
  Calculate potential food production capacity across the city.
\item
  Track progress toward \textbf{30\% local food production goal}.
\item
  Optimize urban farming techniques based on remote sensing data.
\end{enumerate}

\textbf{References}:

Singapore Ministry of Sustainability and the Environment (2021).
``Singapore Green Plan 2030'' (Official government website)
https://www.greenplan.gov.sg/

\begin{center}\rule{0.5\linewidth}{0.5pt}\end{center}

\section{Reflection}\label{reflection-2}

Singapore's application of remote sensing technologies to support its
Green Plan 2030 demonstrates how a land-constrained city-state can
leverage Earth observation data to inform sustainable development
policies. The integration of satellite, aerial, and UAV platforms
provides multi-scale monitoring capabilities that match the precision
requirements of Singapore's urban planning and environmental management
needs.

What I find most interesting about Singapore's approach is how necessity
has driven innovation. With extremely limited land resources and high
vulnerability to climate change, Singapore cannot afford planning
mistakes or ineffective environmental policies. Remote sensing provides
the evidence base needed to optimize every square meter of the urban
landscape for multiple benefits -- from cooling effects to food
production to biodiversity support.

Singapore's coastal monitoring is particularly critical given that
approximately 30\% of the island lies less than 5 meters above sea
level. The combination of satellite-based monitoring and detailed
elevation modeling enables planners to prioritize adaptation measures
where they're most needed, potentially saving billions in infrastructure
costs through targeted interventions.

The city-state's approach to urban heat island mitigation could serve as
a model for other tropical cities. By systematically mapping surface
temperatures against urban morphology and vegetation cover, Singapore
has developed precise guidelines for how much and what type of urban
greenery is needed to achieve specific cooling targets. This data-driven
approach to climate adaptation could be valuable across Southeast Asia,
where many megacities face similar heat challenges. An output of
Singapore's urban greening policy can be seen in Figure, showing the
integration of vegetation into high-density urban areas through
innovative vertical gardens.

\includegraphics[width=0.8\textwidth,height=\textheight]{index_files/mediabag/parkroyal-on-pickeri.jpg}

Vertical gardens on residential buildings in Singapore
\href{https://www.crabintheair.com/wp-content/uploads/2017/02/parkroyal-on-pickering-review-e1537299078683.jpg}{Source}

\bookmarksetup{startatroot}

\chapter{Google Earth Engine}\label{google-earth-engine}

\bookmarksetup{startatroot}

\chapter{Google Earth Engine}\label{google-earth-engine-1}

The week mainly focuses on utilizing Google Earth Engine (GEE) for
remote sensing data processing, covering key functions such as image
filtering, NDVI computation, mosaicking, and spatial analysis for urban
and environmental applications.

\section{Summary}\label{summary-2}

In this week's lecture, we explored Google Earth Engine (GEE) and its
application in processing remote sensing data. GEE is a cloud-based
geospatial analysis platform that allows large-scale processing of
satellite imagery.

Key topics covered in the session included:

\begin{itemize}
\tightlist
\item
  Loading and filtering satellite imagery using Landsat 9 surface
  reflectance data.
\item
  Computing indices such as NDVI (Normalized Difference Vegetation
  Index) to assess vegetation health.
\item
  Mosaicking and clipping images to refine analysis within specific
  study areas.
\item
  Exporting processed images to Google Drive for further analysis.
\end{itemize}

Below is an example of computing NDVI in GEE:

\begin{Shaded}
\begin{Highlighting}[]
\CommentTok{// Load Landsat 9 imagery and filter by date}
\KeywordTok{var}\NormalTok{ dataset }\OperatorTok{=}\NormalTok{ ee}\OperatorTok{.}\FunctionTok{ImageCollection}\NormalTok{(}\StringTok{\textquotesingle{}LANDSAT/LC09/C02/T1\_L2\textquotesingle{}}\NormalTok{)}
    \OperatorTok{.}\FunctionTok{filterDate}\NormalTok{(}\StringTok{\textquotesingle{}2022{-}01{-}01\textquotesingle{}}\OperatorTok{,} \StringTok{\textquotesingle{}2022{-}02{-}01\textquotesingle{}}\NormalTok{)}\OperatorTok{;}

\CommentTok{// Compute NDVI using NIR (Band 5) and Red (Band 4)}
\KeywordTok{var}\NormalTok{ ndvi }\OperatorTok{=}\NormalTok{ dataset}\OperatorTok{.}\FunctionTok{first}\NormalTok{()}\OperatorTok{.}\FunctionTok{normalizedDifference}\NormalTok{([}\StringTok{\textquotesingle{}SR\_B5\textquotesingle{}}\OperatorTok{,} \StringTok{\textquotesingle{}SR\_B4\textquotesingle{}}\NormalTok{])}\OperatorTok{.}\FunctionTok{rename}\NormalTok{(}\StringTok{\textquotesingle{}NDVI\textquotesingle{}}\NormalTok{)}\OperatorTok{;}

\CommentTok{// Display NDVI layer with color visualization}
\BuiltInTok{Map}\OperatorTok{.}\FunctionTok{addLayer}\NormalTok{(ndvi}\OperatorTok{,}\NormalTok{ \{}\DataTypeTok{min}\OperatorTok{:} \DecValTok{0}\OperatorTok{,} \DataTypeTok{max}\OperatorTok{:} \DecValTok{1}\OperatorTok{,} \DataTypeTok{palette}\OperatorTok{:}\NormalTok{ [}\StringTok{\textquotesingle{}blue\textquotesingle{}}\OperatorTok{,} \StringTok{\textquotesingle{}white\textquotesingle{}}\OperatorTok{,} \StringTok{\textquotesingle{}green\textquotesingle{}}\NormalTok{]\}}\OperatorTok{,} \StringTok{\textquotesingle{}NDVI\textquotesingle{}}\NormalTok{)}\OperatorTok{;}
\end{Highlighting}
\end{Shaded}

\includegraphics[width=0.8\textwidth,height=\textheight]{index_files/mediabag/NDVI.jpg}

Generated Normalized Difference Vegetation Index (NDVI) from the
satellite images
\href{https://upload.wikimedia.org/wikipedia/commons/a/ae/NDVI.jpg}{Source:
wikipedia}

This lecture reinforced the importance of cloud computing in geospatial
analysis, making it possible to handle large datasets efficiently.

\begin{center}\rule{0.5\linewidth}{0.5pt}\end{center}

\section{Applications}\label{applications-2}

GEE plays a crucial role in urban studies, environmental monitoring, and
policy analysis, offering scalable solutions for spatial data
processing. It is widely used in land-use classification, climate change
research, and urban heat island analysis.

For example, research by Gorelick et al.~(2017) highlights GEE's
capability in multi-temporal analysis of land cover changes, enabling
researchers to assess urban expansion over time. Similarly, Gong et
al.~(2019) utilized GEE for global land cover mapping, demonstrating its
effectiveness in large-scale remote sensing applications.

In the policy domain, GEE is instrumental in monitoring deforestation
and urban sprawl, assisting government agencies in spatial planning and
disaster management (Hansen et al., 2013). By integrating socio-economic
datasets, GEE can also help in analyzing transport accessibility and
social inequalities, making it a valuable tool in urban planning and
mobility studies.

\includegraphics[width=0.8\textwidth,height=\textheight]{index_files/mediabag/remotesensing-12-016.jpg}

\href{https://www.mdpi.com/remotesensing/remotesensing-12-01655/article_deploy/html/images/remotesensing-12-01655-g007-550.jpg}{Source:MDPI
Remote Sensing}

\textbf{References}:

Gorelick, N., Hancher, M., Dixon, M., Ilyushchenko, S., Thau, D., \&
Moore, R. (2017). Google Earth Engine: Planetary-scale geospatial
analysis for everyone. Remote Sensing of Environment, 202, 18-27.

Gong, P., Li, X., \& Zhang, W. (2019). 40-Year global land cover change
dataset based on Landsat imagery. Science Bulletin, 64(11), 756-763.

Hansen, M. C., Potapov, P. V., Moore, R., Hancher, M., Turubanova, S.
A., Tyukavina, A., et al.~(2013). High-resolution global maps of
21st-century forest cover change. Science, 342(6160), 850-853.

\begin{center}\rule{0.5\linewidth}{0.5pt}\end{center}

\section{Reflection}\label{reflection-3}

This week was my first introduction to Google Earth Engine (GEE), and
it's been an exciting start to exploring how cloud-based platforms can
transform remote sensing analysis. GEE's ability to process large-scale
satellite imagery efficiently is impressive, and it's truly fascinating
that it allows for analysis at a global scale while also enabling a
zoomed-in focus on specific countries or regions. As a beginner, I found
it easy to filter and visualize Landsat data, and the speed of
processing was a big advantage compared to traditional methods. The
hands-on approach of working directly with satellite imagery, like
computing NDVI to analyze vegetation health, was a valuable learning
experience.

However, as much as I'm amazed by GEE's power, this week also made me
reflect critically on the limitations of remote sensing. For instance,
the example involving famine highlighted an important lesson: while
satellite data can provide valuable insights, it can't tell the whole
story. Remote sensing is often most effective when combined with other
contextual data, such as socioeconomic or local environmental
information. This reinforces the point that while tools like GEE are
powerful, they need to be used thoughtfully and in combination with
other data sources to ensure that conclusions are meaningful and useful
for decision-making.

I also realized that, although GEE streamlines complex processes like
mosaicking and clipping, working with satellite imagery still requires
careful attention to detail. There are steps, such as atmospheric
corrections, that need to be carefully considered to ensure data
accuracy. This session has reinforced the idea that tools like GEE are
incredibly useful, but they also require critical thinking and domain
knowledge to interpret the results properly.

In the future, I look forward to diving deeper into GEE and learning how
to leverage its capabilities for more complex urban and environmental
studies. The experience has opened my eyes to the potential of
cloud-based geospatial tools, and I'm excited to apply them in my
research.

\bookmarksetup{startatroot}

\chapter{Classification II}\label{classification-ii}

\bookmarksetup{startatroot}

\chapter{Classification I}\label{classification-i}

The sixth lecture of CASA0023 covers remote sensing classification
techniques, which are crucial for interpreting satellite and aerial
imagery. These methods are used to classify land cover and land use, and
they are key in urban planning, environmental monitoring, and disaster
management.

\section{Summary}\label{summary-3}

Classification assigns pixels in remote sensing images to predefined
categories. The primary classification methods are:

\begin{itemize}
\tightlist
\item
  \textbf{Supervised Classification}: This method requires labeled
  training samples to train the classifier. Common algorithms include
  Maximum Likelihood, Support Vector Machines (SVM), and Random Forest.
\item
  \textbf{Unsupervised Classification}: This technique does not require
  labeled samples. Instead, it clusters the pixels based on their
  spectral properties. Popular algorithms include K-Means and ISODATA.
\end{itemize}

\includegraphics{index_files/mediabag/0-Uzqy-gqZg77Wun0e.webp}

Supervised vs Unsupervised Classification
\href{https://medium.com/@recrosoft.io/supervised-vs-unsupervised-learning-key-differences-cdd46206cdcb}{Source}

\subsection{Supervised Classification}\label{supervised-classification}

Supervised classification involves four main steps:

\begin{itemize}
\item
  \textbf{Training Sample Collection}: Representative areas of different
  land cover types are selected to create labeled training data.
\item
  \textbf{Training the Classifier}: The classifier is trained using
  these samples.
\item
  \textbf{Classifying}: The trained classifier is applied to the entire
  image to assign each pixel a category.
\item
  \textbf{Accuracy Assessment}: The classification result is evaluated
  using metrics like confusion matrices to assess its accuracy.
\end{itemize}

\subsection{Unsupervised
Classification}\label{unsupervised-classification}

Unsupervised classification uses the algorithm to group data based on
pixel similarity, without predefined labels. The steps are:

\begin{itemize}
\item
  \textbf{Defining the Number of Clusters}: A predefined number of
  clusters is set.
\item
  \textbf{Clustering}: The algorithm groups the pixels based on their
  spectral similarity.
\item
  \textbf{Cluster Labeling}: The resulting clusters are labeled based on
  additional knowledge or analysis.
\end{itemize}

\begin{center}\rule{0.5\linewidth}{0.5pt}\end{center}

\section{Applications}\label{applications-3}

Remote sensing classification techniques, particularly supervised and
unsupervised methods, have numerous applications in various research
fields, urban planning, and policy-making. These methods are extensively
used for land use and land cover (LULC) mapping, providing essential
insights into urban growth, environmental changes, and resource
management.

For instance, in urban growth monitoring, remote sensing classification
plays a crucial role in mapping the expansion of cities. A study by Seto
et al.~(2011) applied supervised classification to assess the rate of
urban sprawl in Chinese cities, quantifying the spatial extent of urban
growth and the changes in land use patterns over several decades. This
approach allowed for a better understanding of the environmental
implications of urban expansion and helped policymakers address the
challenges posed by rapid urbanization.

In the context of disaster management, remote sensing classification
techniques are invaluable for assessing damage after natural disasters.
Following Hurricane Katrina, Reif et al.~(2011) used remote sensing data
to classify areas as damaged or undamaged, providing critical
information for recovery efforts. By applying supervised classification
methods, they were able to differentiate between flooded and non-flooded
regions, allowing for more targeted and efficient disaster relief
operations.

Remote sensing classification is also widely used in agriculture,
particularly for monitoring crop health and land productivity. Misbah et
al.~(2022) employed satellite imagery and classification techniques to
analyze crop types and forecast yields in African agricultural land. By
tracking crop conditions over time, remote sensing contributes to
sustainable agricultural practices and informs food security policies,
supporting better decision-making for the future of farming.

These applications highlight the versatility and impact of remote
sensing classification techniques across various sectors, providing
valuable insights for urban planning, environmental management, disaster
response, and agricultural optimization. Furthermore, in the realm of
policy development, remote sensing classification supports land-use
regulations and environmental laws. Governments use these techniques to
monitor urban growth, ensure compliance with zoning policies, and design
green urban spaces. Additionally, it informs sustainable agricultural
policies by analyzing crop types and land productivity.

\textbf{References}:

Seto, K. C., Güneralp, B., \& Hutyra, L. R. (2011). Global forecasts of
urban expansion to 2030 and direct impacts on biodiversity and carbon
pools. Proceedings of the National Academy of Sciences, 109(40),
16083-16088.

Reif, M. K., et al.~(2011). Post-Katrina Land-Cover, Elevation, and
Volume Change Assessment along the South Shore of Lake Pontchartrain,
Louisiana, U.S.A. Journal of Coastal Research, 30(10062), 30--39.

Misbah, K., et al.~(2022). Multi-sensors remote sensing applications for
assessing, monitoring, and mapping NPK content in soil and crops in
African agricultural land. Remote Sensing (Basel, Switzerland), 14(1),
81-.

\begin{center}\rule{0.5\linewidth}{0.5pt}\end{center}

\section{Reflection}\label{reflection-4}

This week's lecture on remote sensing classification techniques provided
a valuable overview of how geospatial data can be leveraged in urban and
environmental research. I gained a solid understanding of the two
primary classification methods---supervised and unsupervised
classification---and how they can be applied in various contexts such as
land cover mapping, disaster assessment, and environmental monitoring.
The practical examples were particularly insightful, as they
demonstrated the real-world utility of these techniques.

What I found most enlightening was the distinction between supervised
and unsupervised classification. Supervised classification, which relies
on labeled training samples, seems ideal when prior knowledge is
available, but I recognize that it can be resource-intensive and may
introduce bias based on the selected training data. On the other hand,
unsupervised classification, which groups data based on inherent
patterns, can be useful when limited prior knowledge exists, but it can
also result in misclassification if the data patterns are too complex or
not well-defined. This has made me think critically about the importance
of choosing the right classification approach based on the dataset and
research context. In future projects, I will need to be mindful of these
trade-offs and select the method that best aligns with the research
question and available data.

Another key takeaway was learning about classification accuracy
assessment techniques, such as confusion matrices. While they are an
essential tool for validating results, I now appreciate that accuracy
assessment alone does not guarantee the quality or interpretability of
the classification. For example, in complex urban environments, even
accurate classifications may not reflect the true complexity of land
cover, especially in rapidly changing or heterogeneous areas. This
reminds me of the importance of not relying solely on quantitative
accuracy but also integrating qualitative insights and domain knowledge
into the interpretation of results.

Looking ahead, I plan to apply these classification techniques in my
research on urban mobility and transportation inequality. For instance,
I could use remote sensing data to analyze how urban sprawl impacts
transportation infrastructure or explore the spatial relationship
between environmental changes and mobility patterns. However, I will
need to be cautious about the limitations of these techniques in
capturing the full complexity of urban systems and the potential biases
inherent in the data.

\bookmarksetup{startatroot}

\chapter{Classification II}\label{classification-ii-1}

\bookmarksetup{startatroot}

\chapter{Classification II}\label{classification-ii-2}

\section{Summary}\label{summary-4}

In this week's lecture, we explored the evolution of data processing
methods in remote sensing, starting with traditional pixel-based
analysis and progressing to more advanced techniques.

\begin{itemize}
\item
  \textbf{Pixel-based Analysis}: Early remote sensing relied on
  pixel-based analysis, where each pixel was classified based on its
  spectral characteristics. While this method was a starting point, it
  had limitations when dealing with complex land surfaces. Each pixel
  was treated independently, failing to account for neighboring spatial
  patterns.
\item
  \textbf{Sub-pixel Analysis}: To improve upon pixel-based methods,
  sub-pixel analysis was introduced. This approach assumes that each
  pixel's spectral reflectance is a mixture of multiple land cover
  types. Spectral Mixture Analysis (SMA) decomposes the spectral
  features within pixels and estimates the proportions of different land
  cover types, making it more effective for analyzing mixed land
  surfaces.
\item
  \textbf{Object-Based Image Analysis (OBIA)}: A significant
  advancement, OBIA groups neighboring pixels into objects, improving
  classification accuracy by considering spatial and spectral
  characteristics together. One common algorithm used in OBIA is the
  SLIC (Simple Linear Iterative Clustering) algorithm, which divides an
  image into superpixels that are more meaningful for analysis.
\end{itemize}

\includegraphics{index_files/mediabag/Comparison-of-tradit.ppm}
Comparison of traditional pixel-based classification vs.~newer
object-based classification for part of the Tiffin River study area
\href{https://www.researchgate.net/figure/Comparison-of-traditional-pixel-based-classification-vs-newer-object-based_fig5_254599984}{Source}

\begin{itemize}
\tightlist
\item
  \textbf{Accuracy Assessment}: To evaluate the performance of
  classification models, several accuracy assessment methods are used:
\end{itemize}

\begin{enumerate}
\def\labelenumi{\arabic{enumi}.}
\item
  Confusion Matrix: A key tool for understanding classification errors.
\item
  ROC Curve Analysis: Provides a graphical representation of model
  performance, with a higher AUROC value indicating better model
  performance.
\item
  Spatial Cross-Validation: Unlike traditional cross-validation, spatial
  cross-validation ensures that training and test sets are spatially
  separated to avoid spatial autocorrelation and ``spatial cheating,''
  which can lead to biased results.
\end{enumerate}

\begin{itemize}
\tightlist
\item
  \textbf{Overfitting Considerations}: A key challenge in machine
  learning is overfitting, where models perform well on training data
  but fail to generalize to new data. To avoid overfitting, it's
  important to: Choose appropriate methods based on data
  characteristics. Select the right number of endmembers for sub-pixel
  analysis. Use multiple accuracy metrics and account for spatial
  autocorrelation.
\end{itemize}

\begin{center}\rule{0.5\linewidth}{0.5pt}\end{center}

\section{Applications}\label{applications-4}

Remote sensing classification techniques have found broad applications
in both public health and biodiversity conservation, enabling more
effective monitoring of disease transmission and environmental changes,
which are crucial for understanding human health risks and preserving
biodiversity.

\subsection{Public Health and
Epidemiology}\label{public-health-and-epidemiology}

Remote sensing classification techniques have become an essential tool
in spatial epidemiology, providing invaluable insights into disease
transmission and environmental health. These techniques allow for the
mapping of population distributions and movement patterns, which are
critical for understanding how diseases spread, as demonstrated by Tatem
et al.~(2017). The potential of remote sensing was particularly evident
during the COVID-19 pandemic, where satellite imagery was used to
analyze urban environmental conditions, map population density, track
transmission risk zones, and explore socio-environmental factors
influencing disease spread.

\subsection{Biodiversity Conservation}\label{biodiversity-conservation}

Remote sensing has also played a key role in biodiversity conservation.
By leveraging advanced classification methods, Hansen et al.~(2013)
mapped global forest cover changes, offering crucial data for
understanding ecosystem dynamics and deforestation patterns. These
techniques have been instrumental in monitoring forest cover and
detecting subtle ecological transformations, as well as in assessing the
impact of environmental changes on biodiversity.

In addition, remote sensing has been widely applied in habitat and
species distribution studies. Leimgruber et al.~(2003) used these
techniques to map critical habitats for endangered species, analyze
landscape fragmentation, and monitor habitat quality. Moreover, these
tools have proven valuable in tracking shifts in vegetation zones,
identifying climate-induced ecosystem changes, and predicting potential
species migration patterns, as highlighted by Horning et al.~(2020).

\textbf{References}:

Tatem, A. J. (2014) Mapping population and pathogen movements.
\emph{International health}. {[}Online{]} 6 (1), 5--11.

Hansen, M. C. et al.~(2013) High-Resolution Global Maps of 21st-Century
Forest Cover Change. \emph{Science (American Association for the
Advancement of Science)}. {[}Online{]} 342 (6160), 850--853.

Leimgruber, P. et al.~(2003) Fragmentation of Asia's remaining
wildlands: implications for Asian elephant conservation. \emph{Animal
conservation}. {[}Online{]} 6 (4), 347--359.

Horning, N. et al.~(2020) \emph{Remote sensing for ecology and
conservation\,: a handbook of techniques / Ned Horning, Julie A.
Robinson, Eleanor J. Sterling, Woody Turner, and Sacha Spector.} Oxford:
Oxford University Press.

\begin{center}\rule{0.5\linewidth}{0.5pt}\end{center}

\section{Reflection}\label{reflection-5}

This week's content gave me a deeper understanding of remote sensing
classification techniques. At first, I was amazed by how powerful these
methods are, especially in helping us understand complex features of the
Earth's surface. When processing remote sensing data, methods like
pixel-based analysis and sub-pixel analysis provided me with many new
insights. However, I also realized that these methods aren't
one-size-fits-all. For example, traditional pixel classification seems
simple but doesn't consider the spatial relationships between pixels,
which is a crucial aspect of natural environments.

Additionally, the introduction of sub-pixel analysis gave me a new
perspective on how to describe land cover types more accurately. By
decomposing the spectral characteristics within each pixel, sub-pixel
analysis helps us extract more detailed information from complex
surfaces. This method made me realize that remote sensing isn't just
about ``processing'' images, but about digging deeper into the
information behind them.

While learning these techniques, I also began reflecting on how to apply
them more effectively. For instance, avoiding overfitting is something I
need to be cautious about. Many times, when applying these methods, it's
easy to be distracted by noise in the data, leading to a model that
performs well on training data but doesn't generalize effectively to new
data. This issue made me realize that behind an accurate model isn't
just complex algorithms, but also a deep understanding of the data
itself. So, in my future work, I will focus more on the validation
process to ensure my analysis is meaningful.

Overall, these classification techniques not only enhanced my
understanding of remote sensing data but also made me think about how to
balance the complexity of the technology with the reliability of the
data in real-world applications. As I become more familiar with these
methods, I'll pay more attention to how to choose and use them wisely to
solve practical problems.

\bookmarksetup{startatroot}

\chapter{Synthetic Aperture Radar
(SAR)}\label{synthetic-aperture-radar-sar}

\bookmarksetup{startatroot}

\chapter{Synthetic Aperture Radar
(SAR)}\label{synthetic-aperture-radar-sar-1}

This week we studied Synthetic Aperture Radar (SAR), a unique remote
sensing system that captures surface information using microwaves.
Unlike optical sensors, SAR provides two key data types - amplitude and
phase - enabling all-weather Earth observation.~

\section{Summary}\label{summary-5}

Synthetic Aperture Radar (SAR) provides unique Earth observation
capabilities through its two fundamental
measurements:~\textbf{amplitude}~(signal strength)
and~\textbf{phase}~(wave cycle position). As illustrated in Figure,
these measurements reveal complementary information - while amplitude
highlights surface texture variations (e.g., bright urban areas versus
dark water bodies), phase data enables millimeter-scale deformation
monitoring through interferometric analysis (InSAR). This dual-data
nature makes SAR exceptionally valuable for applications ranging from
disaster response to infrastructure monitoring.

\includegraphics{index_files/mediabag/Fig-A2-Images-of-amp.png}

Images of amplitude (left) and phase difference (right) for ERS SAR data
spanning the 1999 Hector Mine Earthquake
\href{https://www.researchgate.net/figure/Fig-A2-Images-of-amplitude-left-and-phase-difference-right-for-ERS-SAR-data_fig8_256830372}{Source}

The choice of radar wavelength significantly impacts observation
capabilities, as detailed in Table. Shorter wavelengths like X-band (3
cm) provide detailed surface information but limited penetration, making
them ideal for ice and snow monitoring. The commonly used C-band (5 cm)
offers balanced capabilities for agricultural and disaster monitoring,
while L-band's longer wavelength (23 cm) penetrates vegetation canopies,
enabling forest structure analysis and permafrost studies. This
wavelength-dependent penetration is particularly crucial when selecting
data for specific research needs.

\begin{longtable}[]{@{}llll@{}}
\toprule\noalign{}
Band & Wavelength & Penetration Depth & Best For \\
\midrule\noalign{}
\endhead
\bottomrule\noalign{}
\endlastfoot
X & \textasciitilde3 cm & Leaves & Ice/snow monitoring \\
C & \textasciitilde5 cm & Canopy top & Agriculture, disasters \\
L & \textasciitilde23 cm & Through vegetation & Deforestation,
permafrost \\
\end{longtable}

Three additional factors critically influence SAR interpretation:

\begin{enumerate}
\def\labelenumi{\arabic{enumi}.}
\tightlist
\item
  Polarization configurations determine target interactions - VV
  polarization enhances horizontal features like roads, whereas VH
  reveals volumetric scattering from vegetation.
\item
  Surface dielectric properties affect reflectivity, with moist surfaces
  typically appearing brighter in SAR images.
\item
  Data scaling (power, amplitude, or decibel) determines analytical
  approaches, where: Power scale preserves original values for
  statistical modeling/ Amplitude scale improves visual interpretation/
  dB scaling (logarithmic) enhances contrast for change detection
\end{enumerate}

In Google Earth Engine, these principles become operational through
Sentinel-1's default dB-scaled data, though users can convert to other
scales as needed. The integration of these technical parameters with
practical visual examples (Figures/Tables) creates a comprehensive
foundation for effective SAR analysis across diverse environmental
monitoring applications.

\begin{center}\rule{0.5\linewidth}{0.5pt}\end{center}

\section{Applications}\label{applications-5}

The 2024 review article ``SAR Features and Techniques for Urban
Planning'' published in~Remote Sensing~presents a comprehensive
examination of Synthetic Aperture Radar (SAR) applications in urban
environments. Authored by researchers from the University of Patras, the
study systematically analyzes SAR's operational advantages over optical
remote sensing, particularly its unique capacity for all-weather,
day-night data acquisition and sensitivity to both geometric and
dielectric properties of urban structures. The review emphasizes how
SAR's multi-dimensional data capture - including amplitude (surface
reflectivity), phase (for deformation monitoring via InSAR), and
polarimetric information (for material characterization) - combined with
optical data fusion techniques, significantly enhances urban mapping
accuracy for critical planning applications.

The authors thoroughly document SAR's transformative impact across three
primary urban planning domains. For urban expansion monitoring,
time-series SAR analysis enables precise tracking of informal
settlements and urban sprawl patterns, while advanced texture-based
classification methods improve discrimination of urban features. In
infrastructure management, SAR's millimeter-precision deformation
monitoring capabilities through InSAR technology have proven invaluable
for assessing building stability and underground construction impacts,
with demonstrated success in case studies like Shanghai's metro system.
Perhaps most innovatively, tomographic SAR techniques are
revolutionizing 3D urban modeling by overcoming the layover effects that
traditionally limited radar imaging in dense urban areas, outperforming
LiDAR in consistent all-weather performance.

Emerging technological integrations form a key focus of the review,
particularly the growing synergy between deep learning approaches like
convolutional neural networks and SAR data processing, which is
automating complex urban feature extraction tasks. The authors highlight
promising developments in SAR-IoT sensor fusion for real-time urban
monitoring systems and the expanding role of SAR-derived data in smart
city GIS platforms. However, the review also critically examines
persistent challenges including SAR's inherent data complexity due to
speckle noise and non-intuitive backscatter patterns, the computational
intensity of high-resolution tomographic processing, and the current
lack of standardized analysis frameworks for urban SAR applications.

The methodology followed in this work comprises the three different
tested classification schemes
\href{https://www.researchgate.net/publication/380914201_SAR_Features_and_Techniques_for_Urban_Planning-A_Review}{Source}

\textbf{References}:

Koukiou, G. (2024) SAR Features and Techniques for Urban Planning---A
Review. \emph{Remote sensing (Basel, Switzerland)}. {[}Online{]} 16
(11), 1923-.

\begin{center}\rule{0.5\linewidth}{0.5pt}\end{center}

\section{Reflection}\label{reflection-6}

Although I had learned the basics of SAR in previous courses, diving
deeper into phase information and interferometry has been particularly
fascinating to me. I used to think of SAR as just another way of
``taking pictures,'' so discovering how it can detect centimeter-scale
ground movements through microwave phase shifts completely changed my
understanding of remote sensing.

At first, I struggled with some of these concepts---especially how phase
data translates into surface deformation. But when I finally saw
interferometric SAR (InSAR) results showing actual ground displacement,
that ``aha!'' moment was incredibly rewarding. The fact that this
technology can monitor earthquakes, volcanic activity, and even building
subsidence makes it far more powerful than traditional optical remote
sensing.

However, I've also realized that mastering SAR analysis isn't easy. The
preprocessing steps are complex, interpretation requires experience, and
results can be sensitive to various noise sources. When I tried running
simple case studies in GEE, I noticed the same algorithm could produce
very different results depending on the region---which made me
appreciate why called SAR ``both powerful and finicky.''

Looking ahead, I want to focus on two key areas: First, time-series
analysis, since SAR's biggest advantage is its all-weather monitoring
capability. Second, exploring fusion with optical data to compensate for
the limitations of single-source analysis. Even though my skills are
still developing, every time I successfully process a meaningful SAR
image, the sense of accomplishment is really motivating.

The learning curve for SAR may be steep, but each step forward reveals
new perspectives. Perhaps that's the beauty of remote sensing---we're
not just analyzing data, but learning how to ``see'' invisible changes
in our world.



\end{document}
